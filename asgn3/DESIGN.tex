\documentclass{article}
\usepackage[utf8]{inputenc}
\usepackage[T1]{fontenc}
\usepackage{titling}
\usepackage{amsmath}
\usepackage[Symbol]{upgreek}
\usepackage{algorithm}
\usepackage{listings}
\usepackage{algpseudocode}
\usepackage{xcolor}

\definecolor{codegreen}{rgb}{0,0.6,0}
\definecolor{codegray}{rgb}{0.5,0.5,0.5}
\definecolor{codepurple}{rgb}{0.58,0,0.82}
\definecolor{backcolour}{rgb}{0.95,0.95,0.92}

\lstdefinestyle{mystyle}{
    backgroundcolor=\color{backcolour},   
    commentstyle=\color{codegreen},
    keywordstyle=\color{magenta},
    numberstyle=\tiny\color{codegray},
    stringstyle=\color{codepurple},
    basicstyle=\ttfamily\footnotesize,
    breakatwhitespace=false,         
    breaklines=true,                 
    captionpos=b,                    
    keepspaces=true,                 
    numbers=left,                    
    numbersep=5pt,                  
    showspaces=false,                
    showstringspaces=false,
    showtabs=false,                  
    tabsize=2
}
\lstset{style=mystyle}

\DeclareUnicodeCharacter{2212}{-}
\setlength{\droptitle}{-12em} 
\makeatletter
  \def\title@font{\Large\bfseries}
  \let\ltx@maketitle\@maketitle
  \def\@maketitle{\bgroup%
    \let\ltx@title\@title%
    \def\@title{\resizebox{\textwidth}{}{%
      \mbox{\title@font\ltx@title}%
    }}%
    \ltx@maketitle%
  \egroup}
\title{Assignment 3 Sorting: Putting your affairs in order}
\author{Design Document}
\date{Derrick Ko - Winter 2023}

\begin{document}
\maketitle
\section{Overview}
In this assignment we will be using different sorting algorithms to sort numbers while calculating the stats of amount of elements sorted, moves it took to get sorted and compares between numbers.
\section{batcher.c}
Merge Exchange Sort (Batcher’s Method) in Python
\begin{lstlisting}[language=Python]
def comparator (A: list , x: int , y: int):
    if A[x] > A[y]:
        A[x], A[y] = A[y], A[x]

def batcher_sort (A: list ):
    if len(A) == 0:
        return

    n = len(A)
    t = n. bit_length ()
    p = 1 << (t - 1)

    while p > 0:
    q = 1 << (t - 1)
    r = 0
    d = p

        while d > 0:
            for i in range (0, n - d):
                if (i & p) == r:
                    comparator (A, i, i + d)
                    d = q - p
                    q >>= 1
                    r = p
        
        p >>= 1
\end{lstlisting}\
\\ [4\baselineskip]
\section{shell.c}
Shell Sort in Python
\begin{lstlisting}[language=Python]
 def shell_sort(arr):
    for gap in gaps:
         for i in range (gap, len(arr)):
            j = i
            temp = arr[i]
             while j >= gap and temp < arr[j - gap]:
                 arr[j] = arr[j - gap]
                 j -= gap
             arr[j] = temp
 \end{lstlisting}
\section{heap.c}
Heap maintenance in Python
\begin{lstlisting}[language=Python]
def max_child (A: list , first : int , last : int ):
    left = 2 * first
    right = left + 1
    if right <= last and A[ right - 1] > A[ left - 1]:
        return right
    return left
    
    def fix_heap (A: list , first : int , last : int ):
        found = False
        mother = first
        great = max_child (A, mother , last )
    
    while mother <= last // 2 and not found :
        if A[ mother - 1] < A[ great - 1]:
            A[ mother - 1], A[ great - 1] = A[ great - 1], A[ mother - 1]
            mother = great
            great = max_child (A, mother , last)
        else :
            found = True
\end{lstlisting}
Heapsort in Python
\begin{lstlisting}[language=Python]
 def build_heap (A: list , first : int , last : int):
     for father in range ( last // 2, first - 1, -1):
         fix_heap (A, father , last )

 def heap_sort (A: list ):
     first = 1
     last = len(A)
     build_heap (A, first , last )
     for leaf in range (last , first , -1):
         A[ first - 1], A[ leaf - 1] = A[ leaf - 1], A[ first - 1]
         fix_heap (A, first , leaf - 1)
\end{lstlisting}
\section{quick.c}
Partition in Python
\begin{lstlisting}[language=Python]
def partition (A: list , lo: int , hi: int):
    i = lo - 1
    for j in range (lo , hi):
        if A[j - 1] < A[hi - 1]:
            i += 1
            A[i - 1], A[j - 1] = A[j - 1], A[i - 1]
        A[i], A[hi - 1] = A[hi - 1], A[i]
        return i + 1
\end{lstlisting}
Recursive Quicksort in Python
\begin{lstlisting}[language=Python]
# A recursive helper function for Quicksort .
def quick_sorter (A: list , lo: int , hi: int ):
    if lo < hi:
        p = partition (A, lo , hi)
        quick_sorter (A, lo , p - 1)
        quick_sorter (A, p + 1, hi)

def quick_sort (A: list ):
    quick_sorter (A, 1, len(A))
\end{lstlisting}
\section{sorting.c (Test Harness)}
\begin{itemize}
    \item -a : Employs all sorting algorithms.
\item -h : Enables Heap Sort.
\item -b : Enables Batcher Sort.
\item -s : Enables Shell Sort.
\item -q : Enables Quicksort.
\item -r seed : Set the random seed to seed. The default seed should be 13371453.
\item -n size : Set the array size to size. The default size should be 100.
\item -p elements : Print out elements number of elements from the array. The default number of
elements to print out should be 100. If the size of the array is less than the specified number of
elements to print, print out the entire array and nothing more.
\item -H : Prints out program usage. See reference program for example of what to print.
\end{itemize}
\section{set.c} 
\subsection{Set empty(void)}
This function is used to return an empty set. In this context, an empty set would be a set in which all bits
are equal to 0.
\subsection{Set universal(void)}
This function is used to return a set in which every possible member is part of the set.
Set
\subsection{Set insert(Set s, uint8\textunderscore t x)}
This function inserts x into s. That is, it returns set s with the bit corresponding to x set to 1. Here, the
bit is set using the bit-wise OR operator. The first operand for the OR operation is the set s. The second
operand is value obtained by left shifting 1 by x number of bits.
\subsection{Set remove(Set s, uint8\textunderscore t x)}
This function deletes (removes) x from s. That is, it returns set s with the bit corresponding to x
cleared to 0. Here, the bit is cleared using the bit-wise AND operator. The first operand for the AND
operation is the set s. The second operand is a negation of the number 1 left shifted to the same position
that x would occupy in the set. This means that the bits of the second operand are all 1s except for the
bit at x’s position. The function returns set s after removing x.
\subsection{bool member(Set s, uint8\textunderscore x)}
This function returns a bool indicating the presence of the given value x in the set s. The bit-wise AND
operator is used to determine set membership. The first operand for the AND operation is the set s.
The second operand is the value obtained by left shifting 1 x number of times. If the result of the AND
operation is a non-zero value, then x is a member of s and true is returned to indicate this. false is
returned if the result of the AND operation is 0.
\subsection{Set union(Set s, Set t)}
The union of two sets is a collection of all elements in both sets. Here, to calculate the union of the two
sets s and t, we need to use the OR operator. Only the bits corresponding to members that are equal to
1 in either s or t are in the new set returned by the function.
\subsection{Set intersect(Set s, Set t)}
The intersection of two sets is a collection of elements that are common to both sets. Here, to calculate
the intersection of the two sets s and t, we need to use the AND operator. Only the bits corresponding
tomembers that are equal to 1 in both s and t are in the new set returned by the function.
\subsection{Set difference(Set s, Set t)}
The difference of two sets refers to the elements of set s which are not in set t. In other words, it refers to
the members of set s that are unique to set s. The difference is calculated using the AND operator where
the two operands are set s and the negation of set t. The function then returns the set of elements in s
that are not in t.
\subsection{Set complement(Set s)}
This function is used to return the complement of a given set. By complement we mean that all bits in
the set are flipped using the NOT operator. Thus, the set that is returned contains all the elements of the
universal set U that are not in s and contains none of the elements that are present in s.

\end{document}
