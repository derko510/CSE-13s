\documentclass{article}
\usepackage{graphicx}
\usepackage{color,soul}
\usepackage{titling}
\usepackage[affil-it]{authblk} 
\setlength{\droptitle}{-10em} 
\title{\Large \textbf Assignment 1 Getting Acquainted with UNIX and C}
\author{Writeup}
\date{Derrick Ko - Winter 2023}
\begin{document}
\maketitle
\section{Summary}
This writeup must include the plots
that you produced using your bash script, as well as discussion on which UNIX commands you used to produce each plot and why you chose to use them.
\section{UNIX Commands Used}
I used \hl{tail -n +2} to remove the 1st line of the Monte Carlo output so it doesn't interfere with any of the output when graphing the plot. 
\\ [2\baselineskip] I used \hl{awk} to select which points of the output to print out from another file. In figure 2, I specifically used awk to calculate if the value was 1 or 0 to see if it was in the circle or outside. This command makes it very easy to manipulate the values inside the data files and output the data that I want to work with. 
\\ [2\baselineskip] \hl{set object circle at 0,0 size 1 fc rgb "black"}. This created the circle at 0,0 with a radius of 1. This seemed like the most simple way to make a circle without having to fully plot out points of a circle and it is built into gnuplot.  
\\ [2\baselineskip]I used \hl{cat} to print the data file as an output so it could be piped onto another command. I picked this becuase it is the better option. I thought of using head which works in a similar way but it wouldn't output the whole output like cat would. 
\begin{figure}[htp]
    \centering
    \includegraphics[width=9cm]{figure2.pdf}
\end{figure}
\begin{figure}[htp]
    \centering
    \includegraphics[width=9cm]{figure3.pdf}
\end{figure}
\begin{figure}[htp]
    \centering
    \includegraphics[width=12cm]{inter.pdf}
    \caption{This graph was created by graphing the error of within the cirlce and outside of the circle. The yellow line is both inside and outside graphed together. The black line is within the circle and green is outside of the circle. It is more accurate within the circle since the estimation of pi could be calculated easier. This is only tested on seed one for all 3 data points.}
\end{figure}

\end{document}
